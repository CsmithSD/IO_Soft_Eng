% !TEX root = SystemTemplate.tex
\chapter{User Stories, Backlog and Requirements}
\section{Overview}

This document contains a summary of what we, The Intolerable Optimists, did during the development of our Program Tester program. The Program Tester is designed to aid professors and TAs in grading student programs by allowing users to write test cases to be performed on the student programs and have a summary of how well the student programs did generated.


%The overview should take the form of an executive summary.  Give the reader a feel 
%for the purpose of the document, what is contained in the document, and an idea 
%of the purpose for the system or product. 

% The userstories 
%are provided by the stakeholders.  You will create he backlogs and the requirements, and document here.  
%This chapter should contain 
%details about each of the requirements and how the requirements are or will be 
%satisfied in the design and implementation of the system.

%Below:   list, describe, and define the requirements in this chapter.  
%There could be any number of sub-sections to help provide the necessary level of 
%detail. 





\subsection{Scope}
This document covers our basic user stories and requirements for this program.

% What scope does this document cover?  This document would contain stakeholder information, 
% initial user stories, requirements, proof of concept results, and various research 
% task results. 



\subsection{Purpose of the System}
This Program is meant to be used in an academic setting to grade student programs. 
This will help educators streamline their grading process by taking out the need to 
compile and test each program individually.

% What is the purpose of the system or product? 


\section{ Stakeholder Information}
This products main stakeholders are professors and TA in education.
The program assists them in grading programs that they receive from students. 
This task can be done manually so this program is not critical to their needs but is a 
great convenience as it saves time during the grading process.

% This section would provide the basic description of all of the stakeholders for 
% the project.  Who has an interest in the successful and/or unsuccessful completion 
% of this project? 


\subsection{Customer or End User (Product Owner)}
Charles Parsons is the Product Owner of this project. He is in charge of understanding the requirements set forth by our
customer, Dr. A. Logar, managing the product backlog, keeping the rest of the team up to date on what
needs to be done in an overall sense, and writing a portion of the code that is required.

% Who?  What role will they play in the project?  Will this person or group manage 
% and prioritize the product backlog?  Who will they interact with on the team to 
% drive product backlog priorities if not done directly? 

\subsection{Management or Instructor (Scrum Master)}
Christopher Smith is the Scrum of this project. He is in charge of managing the schedules of other team members
and keeping track of the sprints of the project.

% Who?  What role will they play in the project?  Will the Scrum Master drive the 
% Sprint Meetings? 


\subsection{Investors}
Investors in this project include Dr Logar, who has set the requirements of the project and the specifications for it.

% Are there any?  Who?  What role will they play? 


\subsection{Developers --Testers}
All three of the members of this team will be testing the program to make sure it works as interned.

%Who?  Is there a defined project manager, developer, tester, designer, architect, 
%etc.? 


\section{Business Need}
This program speeds up the process of grading programs for professors in academia, and save unnecessary repetitive work from needing to be done manually. 

%Use this section to define what business need exist and how this software will 
%meet and/or exceed that business need.   

\section{Requirements and Design Constraints}
This program has been designed to be used on a Linux System and does not requires very many system resources to be run. The program will not run on a Windows system in its current state.
%Use this section to discuss what requirements exist that deal with meeting the 
%business need.  These requirements might equate to design constraints which can 
%take the form of system, network, and/or user constraints.  Examples:  Windows 
%Server only, iOS only, slow network constraints, or no offline, local storage capabilities. 


%\subsection{System  Requirements}
%What are they?  How will they impact the potential design?  Are there alternatives? 


%\subsection{Network Requirements}
%What are they? 


\subsection{Development Environment Requirements}
The program has been developed for use on a Linux System.
%What are they?  Is the system supposed to be cross-platform? 


\subsection{Project  Management Methodology}
%The stakeholders might restrict how the project implementation will be managed. 
% There may be constraints on when design meetings will take place.  There might 
%be restrictions on how often progress reports need to be provided and to whom. 
 
\begin{itemize}
\item We are using Trello to keep track of the backlogs and sprint status.%What system will be used to keep track of the backlogs and sprint status?
\item All Parties will have access to the Sprint and Product Backlogs via Trello.%Will all parties have access to the Sprint and Product Backlogs?
\item This project will encompass on sprint.%How many Sprints will encompass this particular project?
\item The Sprint Cycles are one week long.%How long are the Sprint Cycles?
\item We are using GitHub for our source control and there are currently no restrictions for team members on it.%Are there restrictions on source control? 
\end{itemize}

\section{User Stories}
%This section can really be seen as the guts of the document.  This section should 
%be the result of discussions with the stakeholders with regard to the actual functional 
%requirements of the software.  It is the user stories that will be used in the 
%work breakdown structure to build tasks to fill the product backlog for implementation 
%through the sprints.

%This section should contain sub-sections to define and potentially provide a breakdown 
%of larger user stories into smaller user stories. 



\subsection{User Story Sprint 1 \#1}
As a user I want the program to read in the .tst files and test them on the program.

We want to read in .tst file and run them on the compiled program.

\subsubsection{User Story Sprint 1 \#1 Breakdown}
The test suite will need to compile a program on the command line and run it using redirected input and output to use the test file.

\subsection{User Story Sprint 1 \#2} 
As a user I want the program to compile the .cpp file.
We want to be able to compile the student's .cpp file that is in the root directory.


%\subsubsection{User Story \#2 Breakdown}
%User story \#2  .... 

\subsection{User Story Sprint 1 \#3} 
As a user I want the program to store the percentage of pass and failed test cases in the .log file

\subsection{User Story Sprint 1 Breakdown \#3}
A log file will need to be created in the root directory storing all the results of the test suite.

\subsection{User Story Sprint 2 \#1}
Process the programs of an entire class just given the directory.
 
\subsection{User Story Sprint 2 Breakdown\#1}
The test suite will be placed in the root directory. This directory is to be traversed by the test suite to compile and test every .cpp file within the directory or any sub-directory. 

\subsection{User Story Sprint 2 \#2}
Use acceptance tests to pass or fail a program based on outcome.

\subsection{User Story Sprint 2 Breakdown \#2}
Certain tests shall be labeled as acceptance tests upon which a passing grade for the programs will depend. If they do not pass a failing grade will be assigned. If some are passed then the percentage of passed tests is assigned as a grade.

\subsection{User Story Sprint 2 \#3}
Make a log file for each specific student and cumulative file in the root directory.

\subsection{User Story Sprint 2 Breakdown \#3}
An individual log file will be stored in each students sub-directory and a summary file will be placed in the root directory. The summary file will be named CS150.log.

\subsection{User Story Sprint 2 \#4}
Run multiple tests all of which will be stored in a test sub-directory.

\subsection{User Story Sprint 2 Breakdown \#4}
All tests are guaranteed to be stored in a sub-directory called "tests".
This sub-directory will be stored in the root directory. There may be sub-directories within the tests' directory.

\subsection{User Story Sprint 2 \#5}
Automatically generate test files for the test suite to run. Store these tests in the tests sub-directory.

\subsection{User Story Sprint 2 Breakdown \#5}
The test suite will, at the users request, write files of randomly generated integers or floats as specified by the user. The limit of files will also be specified by the user. The number of integers or floats will be a random number from 1 to 200.

\subsection{User Story Sprint 2 \#6}
Create prompts for the user to make choices concerning the random tests without having to use the execution statement.

\subsection{User Story Sprint 2 Breakdown \#6}
Instead of getting all information from the execution of the test suite, prompt the user for information such as  whether they wish to generate test cases and which kind they wish to generate. 

%\section{Research or Proof of Concept Results}
%This section is reserved for the discussion centered on any research that needed 
%to take place before full system design.  The research efforts may have led to 
%the need to actually provide a proof of concept for approval by the stakeholders. 
%The proof of concept might even go to the extent of a user interface design or 
%mockups. 


%\section{Supporting Material}


%This document might contain references or supporting material which should be documented 
%and discussed  either here if approprite or more often in the appendices at the end.  This material may have been provided by %the stakeholders  
%or it may be material garnered from research tasks.

