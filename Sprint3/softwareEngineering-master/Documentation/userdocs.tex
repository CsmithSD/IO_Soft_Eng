% !TEX root = SystemTemplate.tex

\chapter{User Documentation}

The only step to setting up the tester program is to download the source code and run 
make. The application compiles into one executable which may be run from the command line.

\section{Installation Guide}
Steps for proper installation:
\begin{enumerate}
	\item Download the source code package.
	\item Use the provided Makefile to compile the {\it Grade} program.
	\item Copy the {\it Grade} executable into the class directory.
	\item Setup the class directory as shown in Figure~\ref{directorylayout}.
\end{enumerate}

\section{User Guide}
\textbf{Using the {\it Grade} program to grade student programs:}
\begin{enumerate}
	\item Install the {\it Grade} program as described above.
	\item Populate the test\_int and test\_float directories with test and answer files.
	\item Run the {\it Grade} executable. (\$ ./Grade)
	\item In the main menu, choose option 1 (Grade Student Programs).
	\item The program will automatically compile and execute the student programs with the provided test cases as input.
	\item Open the .log file in the root directory for an overview of the class results.
	\item Individual student log files can be found in the student directories.
\end{enumerate}

\noindent \textbf {Using the {\it Grade} program to generate test cases:}
\begin{enumerate}
	\item Install the {\it Grade} program as described above.
	\item Run the {\it Grade} executable (\$ ./Grade).
	\item In the main menu, select option 2 (Generate Test Files).
	\item Provide the information prompted by the program.
	\item Test files will be created in the test\_int and test\_float directories.
\end{enumerate}

%\section{Programmer Manual}
%None