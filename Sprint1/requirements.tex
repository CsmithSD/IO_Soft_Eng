% !TEX root = SystemTemplate.tex
\chapter{User Stories, Backlog and Requirements}
\section{Overview}


This system is for professors to test their students' programs against test data contained in text files. The results of these
tests will then be written to a text file for th user to evaluate the tested programs performance.

 Title: Dr. Logar's User Story
As a professor I want to have a program compile, run, test, and evaluate the tests run on a program submitted to me by 
my students so that I no longer have to run each test case individually and evaluate them by hand.

Below:   list, describe, and define the requirements in this chapter.  
There could be any number of sub-sections to help provide the necessary level of 
detail. 


The progam should compile and run a program against all test cases contained in files ending in ".tst" within the current 
directory or within a subdirectory. In addition, it will evaluate the passes and fails of the program and log this in a text 
document for user review.


\subsection{Scope}
The purpose of this project is to create a program that will run another program against test documents and output the 
results into another file forthe user. The program is specifically geared toward computer science professors for their use in 
grading student programs. This program will compile and run the submitted program imputting the test data. The results, which are saved to a text document, are evaluated and and percentage of pass/fail is also computed and saved in the
document.

What scope does this document cover?  This document would contain stakeholder information, 
initial user stories, requirements, proof of concept results, and various research 
task results. 



\subsection{Purpose of the System}
This system is designed in order to make it easier for professors to grade their students programs. The test files need only be 
in the same directory (or within a subdirectory) as the program to be compiled and run. Also, the output is to be detailed 
enough so that the user knows exactly which test cases passed and which failed.


\section{ Stakeholder Information}


There are only two main stake holders in this project. The first is Dr, Logar, our customer, who naturally expects a working
progam. The second stake holder is this team who greatly desires an A in this course.


\subsection{Customer or End User (Product Owner)}
Dr. Logar is our customer and end user in this case.  

\subsection{Management or Instructor (Scrum Master)}
Cris Smith is our Scrum Master. He plans the time line we are to follow as well as the scrum meetings 


\subsection{Investors}
There are no Investors.


\subsection{Developers --Testers}
There is not set developer(s) or tester(s). We divide up the work as evenly as possible. After running our own tests on our 
code we are sure to test the finished product to ensure integrating the code was successful.


\section{Business Need}
This program is aimed at reducing the amount of time a professor must personally spend on each program by automating
the test phase.

\section{Requirements and Design Constraints}
The main constraint of this program is that it must run on a Linux based operating system and be used to compile C++
programs. Other constraints are that this program does require a basic knowledge of command line compiling.


\subsection{System  Requirements}
The system requirements are that only C++ programs be used with this project on a Linuz machine.


\subsection{Network Requirements}
There are no network requirements as everything can be done offline.


\subsection{Development Environment Requirements}
Our developement enviornment is Linux using simple text editors with g++ compile command.


\subsection{Project  Management Methodology}
The stakeholders might restrict how the project implementation will be managed. 
 There may be constraints on when design meetings will take place.  There might 
be restrictions on how often progress reports need to be provided and to whom. 
 
\begin{itemize}
\item What system will be used to keep track of the backlogs and sprint status?
\item Will all parties have access to the Sprint and Product Backlogs?
\item How many Sprints will encompass this particular project?
\item How long are the Sprint Cycles?
\item Are there restrictions on source control? 
\end{itemize}

\section{User Stories}
 Title: Dr. Logar's User Story
As a professor I want to have a program compile, run, test, and evaluate the tests run on a program submitted to me by 
my students so that I no longer have to run each test case individually and evaluate them by hand.

This section should contain sub-sections to define and potentially provide a breakdown 
of larger user stories into smaller user stories. 



\subsection{User Story \#1}
User story \#1 discussed. 

\subsubsection{User Story \#1 Breakdown}
Does the first user story need some division into smaller, consumable parts by 
the reader?  This does not need to go to the level of actual task definition and 
may not be required. 

\subsection{User Story \#2} 

\subsubsection{User Story \#2 Breakdown}
User story \#2  .... 

\subsection{User Story \#3} 

\subsubsection{User Story \#3 Breakdown}
User story \#3  .... 


\section{Research or Proof of Concept Results}
This section is reserved for the discussion centered on any research that needed 
to take place before full system design.  The research efforts may have led to 
the need to actually provide a proof of concept for approval by the stakeholders. 
 The proof of concept might even go to the extent of a user interface design or 
mockups. 


\section{Supporting Material}


This document might contain references or supporting material which should be documented 
and discussed  either here if approprite or more often in the appendices at the end.  This material may have been provided by the stakeholders  
or it may be material garnered from research tasks.

