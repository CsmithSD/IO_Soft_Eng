% !TEX root = SystemTemplate.tex

\chapter{Overview and concept of operations}

The overview should take the form of an executive summary.  Give the reader a feel 
for the purpose of the document, what is contained in the document, and an idea 
of the purpose for the system or product. 


\section{Scope}
The purpose of this project is to create a program that will run another program against test documents and output the 
results into another file forthe user. The program is specifically geared toward computer science professors for their use in 
grading student programs. This program will compile and run the submitted program imputting the test data. The results, which are saved to a text document, are evaluated and and percentage of pass/fail is also computed and saved in the
document.


\section{Purpose}
This system is designed in order to make it easier for professors to grade their students programs. The test files need only be 
in the same directory (or within a subdirectory) as the program to be compiled and run. Also, the output is to be detailed 
enough so that the user knows exactly which test cases passed and which failed.


\subsection{Program Compiler}
Though rather simple to code this is a mojor component because it depends on which system the program is running as to
how the program is compiled. This program is made to run using the gcc command.

\subsection{Test File Location}
Also a simple but necessary component to our system. The program is going to run every file ending in ".tst" within the same
directory as the progam itself. Therefore, searching the current working directory as well as every subdirectory is very 
important to ensure every test case is run.

\subsection{Test Case Evaluation}
Here is the most critical part of the program. This is where the tests are read in by our program and the results evaluated. 
Onceevaluated the results are stored in a text file for the user to review. 

\section{Systems Goals}
The goals of our system is to successfully compile and run a program with the test files in the directory. Then to output the result of the tests in a text file. 

\section{System Overview and Diagram}


\section{Technologies Overview}


